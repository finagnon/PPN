\documentclass[12 pt]{article}
\usepackage[margin=0.8in]{geometry}
\usepackage{graphicx}
\usepackage{array}
\usepackage{amsmath}
\usepackage{amssymb}
\usepackage{caption}
\usepackage{subcaption}
\usepackage{hyperref}
\hypersetup{pdftex,colorlinks=true,allcolors=blue}
\usepackage{hypcap}
\setcounter{tocdepth}{3}
\graphicspath{{./figs/}}

\begin{document} 
\bibliographystyle{abbrv}

\title{Algorithme}
\author{}
\date{\parbox{\linewidth}{\centering%
  \today\endgraf\bigskip
  \vspace*{2 in}
\vspace*{9 cm}
{\centering
\hspace*{-7.5 cm}Encadrant\hspace* {3cm}Dr. Matei Istoan \endgraf\medskip
%\hspace*{-4.5 cm}Git Repo\hspace* {3.5cm}https://github.com/jctdrs/PPN \endgraf\medskip}}}
\maketitle
\newpage
\tableofcontents
\listoffigures
\newpage









\section{Algorithme de division }
\quad Le but de cet algorithme c'est de pouvoir \'{e}ffectuer la division tout en \'{e}chappant au multiples cas d'exceptions ou r\'{e}duire d'\'{e}norme \'{e}cart d'erreur qu'il peut y avoir au cours de l'op\'{e}ration. Pour ce faire nous voudrions stock\'{e} les bits dans un Tableau ou nous pourrons acceder bit par bit.
\qquad Il s'agira de faire une soustraction successive du diviseur de la dividende .
Le Q format ici est le nombre de bits fractionnaire.\\

Algorithme : \\
\\
Variable : numerateur [nnombre de bits ] : Entier,\\ Resultat[taille numerateur + denominateur] : Entier;\\ 


Variable : denominateur[nomre de bits] : Entier, a : reel, b : reel,\\


$Q_{numerateur}$, $Q_{denominateur}$;\\
\\
Debut :

a $\leftarrow a*2^{(Q_{numerateur} + Q_{denominateur})}$;\\

b $\leftarrow b*2^{(Q_{numerateur} + Q_{denominateur})}$;\\

 Si (ab$<$b) Alors\\

\quad \quad Tant-que (a$<$b) Faire \\

\qquad \qquad a $\leftarrow a*2$\\

\qquad \qquad  Resultat[i] = 0;\\

\qquad \qquad i++;\\



 \qquad FinTantQue\\

SiNon \\






\qquad \qquad TantQue a-b $>$ 0 Alors\\

\qquad \qquad \qquad Resulta[i] = 1 ;\\

\qquad \qquad FinTantQue\\



FinSi \\

Si Resultat[0] = 0;\\


\qquad Pour i \in [0, tailleBit];

\qquad \qquad \qquad Resultat$ = 0. Resulta[i]$;

SiNon\\


\qquad Pour i \in [0, tailleBit];\\


 \qquad \qquad \qquad Resultat = Resultat[i];\\

 
FinSi\\

 Lire(resultat);\\
Fin



 




\end{document}\\


